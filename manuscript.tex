\documentclass{sokuten}

\usepackage{luatex85,luatexja}
\usepackage{graphicx, xcolor}
\usepackage{textcomp}
\usepackage[unicode]{hyperref}
\usepackage{fancyvrb}
\usepackage{fvextra}
\usepackage{listings}
\usepackage{siunitx}

% 欧文組版のマイクロタイポグラフィー(細かな読み易さの調整)を有効化
\usepackage{microtype}

% フォントのエンコード
\usepackage[T1]{fontenc}

% OTF フォント(和文フォント)
\usepackage{luacode}
\usepackage{luatexja-ruby}
\usepackage{luatexja-otf}
% \usepackage[noembed,deluxe,jis2004]{luatexja-preset} % 源ノ角/源ノ明朝フォント

% フォント設定(数式)
\usepackage[math-style=ISO,bold-style=ISO]{unicode-math}

\title{薬罐が美しい}
\author{Giraffe Beer}
\volume{3}
\date{}

\begin{document}
% \pagestyle{empty}
\maketitle

\section{はじめに}
薬罐を買った。
2024年も終わろうという時期に薬罐である。
昭和のマンガでしか見たことのない薬罐である。
しかもコーヒーケトルとか気の利いたハイカラなやつでもなく、湯が沸くとピーと鳴く、あのステレオタイプな薬罐である。
そしてわたしはこれを大層気に入っている。せっかくなので訳を記そう。

\section{あらまし}
わたしはコーヒーを消費する。ボダムのフレンチプレスは埃をかぶって滅多に日の目を見ず、もっぱら激安のインスタントを毎日何杯か飲む。
水分補給や気分転換のための嗜好品としてはもちろん、頭痛予防のためのカフェイン補給の意味が強く、味や香りに特にこだわりはない。

とはいっても最低限のラインは存在し、特に粉の溶け残りが好ましくないのである。
カップにインスタント粉を計り、水道水をゆっくり注ぎながらマドラーで攪拌し、電子レンジで温めるというプロセスで一杯を淹れているのだが、蛇口の水勢を慎重に絞らないと粉が塊のまま巻き上げられて溶け残る。
料理用語でいうところのダマになるという現象である。その際に生じる泡もあまり好ましいとは言い難い。茶色の泡がカップ内壁にこびりついて乾いたさまは見苦しいのだ。

そんな折、ふと熱湯をコーヒー粉末の上へ注いだところ、これが大変良かった。沸騰した湯は褐色のパウダーを瞬時に溶かしきり、ダマも泡も残さず均一な液体へと変貌させる。口当たりはなめらかで、溶け残りの粉っぽさもなく、マドラーも邪魔にならない。においの元となる化合物の揮発も激しくなるためか、心なしか香りも良い。
考えてみればコーヒーは溶液であり、溶媒の温度は高いほうが良いに決まっている。そんなわけで、しばらくは台所へ赴き、ダイソーの雪平鍋で湯を沸かしてはコーヒー片手に帰ってくる生活が続いていた。

\section{改善}
しかししばらく使っていれば不満点が見えてくる。
雪平鍋の使い勝手がそこまで良くないのだ。長いハンドルは衣類を引っ掛ける恐れがあり、申し分程度の注ぎ口からは湯が左右にこぼれる。湯が沸いたことを通知する機能もないため、コンロにかけたことを忘れて空焚きしたことも何度かあった。調理に重きを置いた汎用鍋なのだから当然ではあるが、コーヒー作成にしか使わない運用に最適であるとはいいがたい。
なによりフタが存在せず、水面から熱が逃げ放題であり、投入したエネルギーつまりガス代に対する湯沸かし効率が悪い。効率の悪い行為を漫然と続けるのはどうにも気になってしまうのだ。
では別の湯沸かし手段はどうか?
\section{コンペティション}
いわゆる電気ケトル、つまり底に電熱線のついた水コンテナは確かに要求性能を満たしている。コンパクトで、密閉されており、湯が沸けばサーモスタットが働いて勝手に加熱を停止する。
が、電気代の高騰が叫ばれる昨今に毎日何度も電力を熱に変えるというのは、どうにも気分的に落ち着かないものがある。実際に算出してみれば微々たる金額であったとしてもだ。安いものであれば2千円を切るが、されど2千円。気の乗らなさで購入の手を鈍らすには充分な金額なのだ。

一般的に大量生産は効率を上げ単価を下げる。これは湯にも適用されそうだ。大量の湯は高い比熱をもち冷めにくくなる。
では電気ケトルの容積を上げ、側面を真空二重の断熱構造にして保温性を確保、注ぎ口も外気の侵入を防ぐため密閉、水蒸気による破裂を防ぐため逆止排気弁をつけて、持ち上げてカップへ注ぐにはケトル全体が重くなるであろうからポンプによる排水を、と考えていくと、一般的に電気ポットと呼ばれるものにたどり着く。いつでも熱湯が出る利便性を抜きにしても、構造としてとても理に適っている。昭和の時代から大抵のご家庭に鎮座しているのも頷けるというものだ。

しかし先述のごとくほいほいとユニットを積み足していくと、当然その恩恵はコストに跳ね返ってくる。どんなに安いものでも5千円を超え、平均的には1万円の大台に乗ってくる相場感だ。いくら効率がよいといっても、これでは浮いたランニングコストを回収する前に製品寿命が来かねない。なによりポットでしか達成できない目標があるわけでもなく、湯沸かしの用は既に雪平鍋で足りているのだ。この出費は流石に許容できない。

と考えていき、ユニット増がコスト増につながる点からふと起点に立ち戻る。なにも熱源を新しく購入する必要はないのだ。熱源なら既にガスコンロという強力で使い勝手のいいものが存在する。それを水に伝達する鍋の部分だけを改善すればいい話だ。
\section{薬罐}
という運びで薬罐を購入した。
南部鉄瓶やコーヒーケトルといったハイカラなものではなく、\num{1190}円の、平底に金属半球をのせて取っ手と注ぎ口をつけただけの極めてシンプルなものだ。ぶっきらぼうと言ってもいい。
そのシンプルさから来る安さを売りとしているモデルなので、前述の見えている構造が全てである。そしてそのシンプルな構造が、人間のために湯を沸かすという目的をなにひとつ過不足なく達成している。構造のすべてにおいてコンセプトが一貫しているのだ。折角なのでひとつひとつ見ていこう。

まず底面。直径\SI{175}{mm}の真円の底は一枚の金属板で構成されている。同心円状につけられた窪みはリブ加工だ。V字に折った紙が折り目方向にたわまないのをイメージしていただくと良いが、同じ厚みの板でも曲げ線を入れるだけで強度が桁違いに増す。補強材を入れる、厚みを増やすなどのコストのかかる方法と違い、型を高圧で押し付けるプレス加工一発で済むのが利点である。シンプルかつ低コストに目的を達成するという製品コンセプトが早速伺えてくる。

その上に乗る金属半球。湯を保持する容器部分である。ここは底面とは対照的に、リブであったり飾りであったり、そういったものが一切ない完全に平滑な球面だ。これもコンセプトから考えると納得がいく。球という構造は非常に強度が高いのだ。補強を入れるまでもなく必要十分な強度が確保される。顔が映るほど磨き上げられたそのステンレスの地肌からは、耐熱塗装費用すら惜しみたいという意思が感じられる。
側面に継ぎ目の一切ない完全な椀状である点、また内側に入っている細かい線からしてこれはスピニング加工であろう。金属円盤を回転させ、ローラーで型に押し付けて変形させる手法だ。金属板がぬるりと変形して球面を構成するさまはなかなかに奇妙な光景であるので、是非ともYouTubeあたりで検索して見ていただきたい。
これは直感的に工数が多く高コストであるように思える。しかし、深さ\SI{150}{\milli\m}には達そうかという奥まった形状でありプレスで作るのは難しい。プレスとは金属を引き延ばす行為なので、当然延ばすに従って厚みは減じていく。平板をドーム状になるほど延ばせばおそらく千切れてしまうだろう。また鋳造よりは遥かに容易なこと、スピニング加工自体鍋やボンベ等安価な日用品によく使われていること、メタ的ではあるがこの薬罐がAmazonで最安の部類として売られていることなどを考えると、想像よりはずっと安く上がるのであろう。

底板があり、その上に乗る容器があるとなれば、次はその2つを接線で接合する必要が生じる。この接合線は水が漏れないよう完全に密閉する必要があるのは当然、さらに熱による金属の膨張にも耐える必要がある。直火の当たる底面は容器に比べて温度が上がり、大きく歪むと考えられる。それによって接合が緩んだらことだ。この薬罐全体において最も強度が求められる部分といえる。
そのためここにはシーミングと呼ばれる接合法が用いられている。金属のフチ同士を噛み合わせ、互いにつぶすことによって繋ぐ手法だ。接合に資材そのものを使うため、溶接や耐熱パッキンよりも安価に水密構造が得られ、そして十分な強度をもつ。
熱による膨張を考えていくと、先程の底面に施されたプレスリブに思い当たる。板面に対して斜めの面が確保されており、ここはバネのように金属の膨張を吸収してくれそうだ。やはり実に合理的な構造といえよう。

半球である容器の頭頂部は\SI{75}{\milli\m}ほどの直径で水平に切り落とされている。注水口としての利用はもちろん、薬罐は調理器具であるため、スポンジと手を突っ込んで内部を洗える必要がある。ちょうど拳が入る大きさの穴であり、使用感として申し分ないサイズだ。
当然穴があきっぱなしでは熱を逃がしてしまうため、封じ込めるために蓋がついている。半球に蓋が乗り、これで水は熱の逃げ場なく温められることとなる。湯から熱を受けた空気は半球の内部に滞留し続け、大気中へのさらなるエネルギー放散を防ぎ、湯沸かしの効率を高める。雪平鍋では実現できなかったきわめて理想的な構造だ。

そしてこの蓋のはまり具合がまた絶妙なのだ。蓋裏に設けられた高さ\SI{10}{\milli\m}程度の嵌めあいリングを薬罐内に挿入して固定する形となるが、そのリングに設けられた段差がある程度のスナップ感・密閉性を提供しており、軽く引っ張った程度では蓋が外れないようになっている。
これは蓋側を密閉することで薬罐内をチャンバーとして圧力を高め、水蒸気を注ぎ口へ導くことで笛を駆動する設計であろう。また瓶の口のようなねじ込みを採用しないのは、コストの面もあろうが、万が一にも笛が詰まるなどして圧力の逃げ場がなくなった際、安全弁として蓋を飛ばすことで容器破裂などより危険な事態を防ぐためのフェイルセーフ構造であると考えられる。何気なく湯を沸かすだけでも、水蒸気は簡単に危険な高圧を生じうることを忘れてはならないのだ。

蓋をまたぐようにつけられた逆U字型の取っ手は真上へと直立し、一切側面へとはみ出さない。省スペース性と安全対策の両立だ。必要とあらば片側へ90度倒せる構造になっており、注水口へのアクセスを確保している。取っ手自体は金具を耐熱プラスチックで覆っただけの簡素な構造だが、熱い薬罐を素手で持ち上げるという目的を完璧に果たしている。ここでもコストカットと目的の両立というコンセプトが実現されているわけだ。

そして笛。注ぎ口の蓋としてつけられた跳ね上げ式の笛。
この薬罐の構造において、これがもっともアイコニックにして優れている点といえよう。駄菓子のフエラムネ同様穴あき板を2枚重ねた構造で、気流が通過する際に生じるカルマン渦の振動により音を発する笛だ。実にシンプル。プレス金属板2枚で作れるためコストも安そうだ。
この笛を水蒸気で駆動するという発想が素晴らしい。
最終的に達成すべき目的は湯が沸いたことの人間への通知だ。であればまずは水が\SI{100}{\degreeCelsius}に達したことを検知する必要がある。水温を計るための温度計、サーモスタット、電子制御された温度センサ、中途半端に工学をかじった現代人としてはさまざまな手段を考えてしまうが、先人はずっと包括的に全体を観測している。水が\SI{100}{\degreeCelsius}になり、沸騰すればかならず蒸気が生ずるのだ。であれば水蒸気、もっと言えば圧力の上昇を監視すればよい。
電子圧力センサ、一定圧力で変形するバルーン状スイッチ、タービン、いろいろと考えられる。が、圧力は平衡を保とうとする、つまり沸騰して圧力が高まったチャンバー内から1気圧の外気へは必ず一方向の気流が生じるのだ。そこに笛をつければ音が鳴る。たいていの人間は音を感知できる。湯が沸いたことを通知できる。
かくして水温の監視と通知という一見複雑な課題が、きわめてシンプルかつローコストな構造で一挙に解決されるわけだ。動作ガスを水自身から調達する点、動作エネルギーはガスコンロ、火という汎用的な熱源によって供給される点も素晴らしい。規格に合ったコネクタや厳密に制御された電圧でしか動作しない電子機器とは違い、薬罐の笛は屋外で起こした火ででも鳴るのだ。

できる限り安く、高効率に湯を沸かし、それを人間へ通知する。その一貫したコンセプトが、少ない数のパーツひとつひとつにこれでもかと適用されている。シンプルゆえの美しさ、機能性のみを追求した格好良さが実体をもってそこにあるのだ。しかもこいつはコーヒーを作ってくれる。わたしの頭痛を和らげてくれる。

そんなわけで我が家の台所には、ちょっとレトロSFめいた素っ気ない銀色の半球が鎮座する運びとなったのだ。
今日もあれはピイと鳴く。湯が沸いたぞとピイと鳴く。
\end{document}
